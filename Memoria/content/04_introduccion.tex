\chapter{Introducción y prefacio}

La Formula Student es una competición internacional que desafía a estudiantes de ingeniería de todo el mundo a diseñar, construir y competir con monoplazas diseñados, construidos y pilotados por los mismos estudiantes. Esta competición proporciona una plataforma excepcional para que los futuros ingenieros pongan en práctica sus conocimientos y adquieran experiencia práctica en todos los ámbitos de la ingeniería, desde la manufactura hasta la gestión de proyectos. El énfasis en la innovación, la eficiencia y la adaptabilidad aporta un valor incalculable a la formación de los jóvenes ingenieros.

Dentro de los puntos clave de la Formula Student durante los últimos años, se encuentra el interés creciente en la tracción eléctrica. En un mundo cada vez más preocupado por la sostenibilidad y la eficiencia energética, los motores eléctricos han surgido como una opción atractiva para la propulsión de vehículos cada vez más grandes. Este cambio de paradigma plantea la cuestión fundamental de cómo diseñar y controlar eficazmente estos sistemas eléctricos para lograr un alto rendimiento sin dejar de ser accesibles por el grueso de la población. La optimización de costes no suele ser un reto en los deportes de motor, incluida la Formula Student, pero la innovación que se lleva a cabo en estos contextos de libertad absoluta permite traer ideas de las competiciones a los vehículos de calle. En este contexto se establece un puente fundamental entre el presente y el futuro de la movilidad sostenible, explorando los elementos técnicos que impulsan el rendimiento de los motores eléctricos y su control, y, al hacerlo, contribuye a dar forma al panorama de la movilidad del mañana. Además, la tracción eléctrica pura no es la única rama de la industria beneficiada por ingenieros conocedores de estos sistemas, también los trenes de potencia híbridos, los generadores en centrales energéticas, incluso los sistemas de gestión de la energía en hogares pueden volverse mucho más eficientes gracias a la investigación y el conocimiento en baterías, motores eléctricos, electrónica de potencia e integración.

Este proyecto se sitúa en el corazón de esta revolución en los deportes de motor, donde la ingeniería se fusiona con la sostenibilidad y la competición para forjar una nueva generación de soluciones de tracción eléctrica. A través de un enfoque riguroso en el diseño y control de motores eléctricos y controladoras, este proyecto busca avanzar en el conocimiento y la aplicación de tecnologías de vanguardia, contribuyendo así a la formación de ingenieros y al desarrollo de soluciones de movilidad más ecológicas.

El desarrollo de sistemas de control para motores eléctricos no es una tarea trivial y demanda un profundo conocimiento de las características del motor, la teoría del control y la programación de dispositivos electrónicos. Además, este último reto se ve incrementado cuando se trata de aplicaciones de bajo coste, pues los dispositivos que controlen esos inversores, generalmente microcontroladores, vendrán muy limitados en cuanto a prestaciones. Por tanto, es necesario optimizar los algoritmos de control, desde el diseño de los mismos sobre el papel hasta la implementación en líneas de código.

Además, la electrónica de potencia está viviendo grandes avances para aplicaciones de todos los rangos de potencia. Los dispositivos semiconductores de carburo de silicio (SiC) están permitiendo mayores densidades de potencia en los rangos de centenares de \textit{watts} hasta rozar el \textit{megawatt}, debido a la reducción de pérdidas y la alta conductividad térmica del material en comparación con sus equivalentes de silicio tradicional. Por otra parte, los dispositivos de nitruro de galio (GaN) consiguen miniaturizar convertidores del rango de unos pocos \textit{watts} hasta casi el \textit{kilowatt} de la misma manera. Otros avances en componentes, como la tecnología de condensadores de film, aceleran más este proceso de miniaturización, que hace posible conseguir cada vez densidades de potencia más elevadas. Junto a estos dispositivos de nueva generación, avances en las técnicas de control y modulación permiten reducir las pérdidas y aumentar así la eficiencia, reduciendo de esa manera los requisitos de gestión térmica y permitiendo diseñar convertidores todavía más pequeños.

En este trabajo se explorará el diseño y control de un inversor trifásico dual de alto rendimiento para motores PMSM, aplicado al entorno de la Formula Student. Se abordarán los aspectos técnicos y prácticos de este proyecto, desde la selección de componentes hasta la implementación de algoritmos de control avanzados.

En particular, el diseño de esta controladora tendrá en cuenta específicamente las necesidades de e-Tech Racing, el equipo de Formula Student de la UPC EEBE. Desde su fundación en 2013, este equipo ha construido monoplazas año tras año para competir en los eventos de España, República Checa, Italia, Holanda y Alemania.

Tras cinco años de evolución de la misma plataforma, el equipo se encuentra diseñando y construyendo un nuevo concepto en el momento en el que se redacta este trabajo. El cambio principal son justamente los motores eléctricos, cuyo nuevo diseño permitirá embeberlos en las propias ruedas del monoplaza debido a su compacticidad, liberando así mucho espacio del chasis. Se usarán dos motores, uno para cada rueda trasera, aunque el plan a largo plazo es implementar otros dos motores en las ruedas delanteras. Para controlarlos se usarán dos inversores Bamocar D3 700-400 de la empresa alemana Unitek. Estos inversores están extremadamente sobredimensionados en potencia, y ocuparán un espacio considerable dentro del chasis. Además, existe cierta limitación en los parámetros modificables, y eso impide programar un control óptimo para el motor.

Para que el inversor sea fácilmente integrable con los futuros monoplazas del equipo necesita contar con algunos componentes usados actualmente por el equipo. Por ejemplo, se usará el mismo microcontrolador que para el resto de ECUs, los mismos conectores de potencia y comunicación, y PCBs de estilos similares al resto de circuitos del monoplaza para que los fabricantes que colaboran con el equipo puedan fabricar todos los circuitos impresos del inversor. Esto permitirá una integración fluida del inversor como una ECU más del monoplaza, con el añadido de la electrónica de potencia personalizada.