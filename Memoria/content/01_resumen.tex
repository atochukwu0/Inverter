\chapter*{Resumen}
Este Trabajo de Final de Grado se enfoca en el diseño y prototipado de un inversor trifásico dual de 80 kW (2x40 kW) y 600 V con control vectorial (FOC) para PMSMs (motores síncronos de imanes permanentes). Este inversor bidireccional busca ser una solución compacta y de alto rendimiento para aplicaciones de tracción eléctrica en vehículos de Formula Student.

El proyecto tiene la intención de alcanzar altas densidades de potencia, lo que implica la implementación de tecnologías de vanguardia como los semiconductores de carburo de silicio (SiC), el uso de materiales compuestos y la aplicación de técnicas de fabricación aditiva. Se trabajará para alcanzar estos objetivos mediante un diseño que tenga en cuenta la correcta gestión térmica, la disposición eficiente de los componentes, la selección adecuada de cableado y conectores, entre otros aspectos fundamentales.

Además, el código del inversor implementa un lazo de control de par que permite trabajar en la región de debilitamiento de campo, además de optimizar el paso por la zona de baja fuerza contraelectromotriz. El control eficiente de la máquina eléctrica permite extraer al máximo toda su potencia en un rango muy grande de velocidades, llegando de forma segura a sus límites.

\chapter*{Abstract}
This Degree Thesis focuses on the design and prototyping of a dual three-phase inverter of 80 kW (2x40 kW) and 600 V with field-oriented vector control (FOC) for PMSMs (Permanent Magnet Synchronous Motors). This bidirectional inverter seeks to be a compact, high-performance solution for electric traction applications in Formula Student vehicles.

The project intends to achieve high power densities, which involves the implementation of cutting-edge technologies such as silicon carbide (SiC) semiconductors, the use of composite materials and the application of additive manufacturing techniques. Work will be done to achieve these objectives through a design that takes into account correct thermal management, the efficient arrangement of components, the appropriate selection of wiring and connectors, among other fundamental aspects.

In addition, the inverter code implements a torque control loop that allows working in the field weakening region, in addition to optimizing the trajectory through the low back electromotive force zone. The efficient control of the electric machine allows it to extract all its power to the maximum in a very wide range of speeds, safely reaching its limits.