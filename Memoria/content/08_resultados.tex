\chapter{Resultados}

\section{Validación}
La validación del sistema es una etapa crítica en el proceso de desarrollo de cualquier proyecto, especialmente en un proyecto de la magnitud y complejidad del convertidor diseñado. La validación se encarga de verificar que todos los componentes y subsistemas del convertidor funcionen correctamente y cumplan con los requisitos y especificaciones establecidos previamente.

Dada la naturaleza multifacética del convertidor, que abarca desde aspectos eléctricos y electrónicos hasta aspectos de control y software, es esencial planificar una estrategia de validación exhaustiva y efectiva. En este sentido, se ha adoptado un enfoque metodológico basado en el modelo en V, que permite validar los subsistemas desde los más específicos hasta los más generales, siguiendo un flujo lógico y sistemático. 

En este capítulo se desarrolla la parte derecha de la 'V', diseñando y ejecutando las pruebas a varios niveles. La estrategia de validación se divide en tres grandes bloques: validación de \textit{hardware}, validación de \textit{software} e integración. Cada bloque tiene sus propios desafíos y requisitos específicos, pero todos contribuyen al objetivo final de asegurar el funcionamiento correcto y confiable del convertidor.

\subsection{Validación de \textit{hardware}}
La parte más crítica en la validación del convertidor es la de \textit{hardware} puesto que cada iteración cuesta tiempo y dinero. Por ello, son las pruebas de \textit{hardware} las que se diseñan y ejecutan más meticulosamente.

\subsection{Validación de \textit{firmware}}

\subsection{Integración}
